\documentclass[11pt]{article}% uses letterpaper by default

%---------- Uncomment one of them ------------------------------
\usepackage[includeheadfoot, top=1in, bottom=1in, hmargin=1in]{geometry}

% \usepackage[a5paper, landscape, twocolumn, twoside,
%    left=2cm, hmarginratio=2:1, includemp, marginparwidth=43pt, 
%    bottom=1cm, foot=.7cm, includefoot, textheight=11cm, heightrounded,
%    columnsep=1cm, dvips,  verbose]{geometry}
%---------------------------------------------------------------
\usepackage{fancyhdr}
\usepackage{verbatim}
\usepackage{url}
\pagestyle{fancy}
\usepackage{graphicx}
\usepackage{enumerate}
\usepackage{setspace}
%\doublespacing
\singlespacing
%\onehalfspacing
%\newcommand{\exercisename}{}

%\rhead{Wednesdays 7-10pm}
\chead{The Sun Lab - May 19, 2021}
%\lhead{Beyond the Solar System}
\renewcommand{\rightmark}{}
%\lfoot{Jenna Lemonias} \cfoot{\thepage} \rfoot{Spring 2011}

\begin{document}
\begin{center}
\huge Lab 6: The Sun
\end{center}
	

\section*{Part 1: Videos}
We'll watch a series of videos (they are all listed here: {\tt http://www.pbs.org/wgbh/nova/labs/videos/\#sun}) introducing solar science and the Helioviewer. As you watch, discuss with your partner(s) and answer these questions in your lab notebook. 
\begin{enumerate}
\item How long does it take a photon to travel from the core of the sun, where it's produced, to the surface?

[On average, about 100,000 years.]

\item Describe the 3 forces that are most relevant in the sun.

[Nuclear Fusion, Gravitation, and Magnetic]

\item How often does the Sun's magnetic north and south flip, in a process called magnetic realignment?

[Every $\sim$11 years.]

\item Does the differential rotation in the Sun's plasma strengthen or weaken the Sun's field lines? How so?

[What’s more, all the stress and strain generated by these two forces actually strengthens the magnetic field, rather than weakening it.

Imagine a spring as the magnetic field lines. The magnetic field inside the Sun is amplified, is strengthened by the rotating motions, and the shearing motions, and the churning motions inside the Sun. It wants to expand upwards, and it does, until it pokes out through the surface of the Sun.]


\item Name two events that can be caused by a magnetic reconnection.

[Solar flares and coronal mass ejections]

\item Why do sunspots look darker than the rest of the sun?

[Because it is relatively cooler than its surroundings. It’s cool because the magnetic fields are so strong that they’re suppressing the flow of heat from below.]

\item What protects the Earth from solar flares?

[The Sun sends storms out in every direction, and Earth is small and far away. Additionally, Earth’s thick atmosphere provides some defense, scattering and absorbing solar particles before they reach the surface. But, our planet also has a secret weapon-- a strong magnetic field it projects into space, mostly generated by molten iron alloys moving in Earth’s outer core.]

\item What causes auroras?

[charged particles from the Sun collide with nitrogen and oxygen molecules in our atmosphere]

\item Why are we humans more vulnerable to exceptionally big solar storms than we were in 1859?

[With societies so reliant on modern technology, the damage would be far worse. Blown transformers could take months to repair, leaving millions without electricity. And the damage would not be limited to our power grid.]

\item Is the following statement true or false? ``The longer its wavelength, the more energy light carries.'' Explain why.

[This distance between one crest and another is known as the light’s “wavelength.” And the shorter it is, the more energy the light carries. Shorter wavelength light has a higher frequency, and frequency scales with energy (one possible answer).]

\item Describe the instruments on the SDO (AIA, HMI).

[The first is AIA, a battery of four telescopes that can look at ten different wavelengths of light. These range from the Sun’s surface up to the highest reaches of the super-hot corona, the key to modeling space weather.]

[The second SDO instrument featured in the lab is called HMI. The “H” stands for “helioseismology,” because it uses sound waves moving through the surface to model changing magnetic fields generated in the convective zone below. By mapping these complex fields across the entire Sun, HMI helps scientists more quickly spot the conditions that can lead to solar storms.]

\item Why is it useful for solar research to have instruments that can look at wavelengths besides the visible that we can see with our eyes?

[Looking at different wavelengths allows scientists to investigate different aspects of the Sun.]

\item If you want to look at a hotter part of the solar atmosphere, do you want to look at smaller or larger wavelength light?

[Smaller wavelength for hotter part (greater energy)]
\end{enumerate}

\section*{Part 2: Solar Cycle}
Click on the Solar Cycle tab on this page and follow the instructions while answering the following questions: {\tt http://www.pbs.org/wgbh/nova/labs/lab/sun/research}

\begin{equation}
    R = k(10g+s)
\end{equation}
where R is the sunspot number, k is the scaling factor/viewing conditions, g is the number of sunspot groups, and s is the number of individual sunspots

\begin{enumerate}
\item Make a table and record your estimates of g, s, and R, and the ``scientific estimates.''
\begin{itemize}
    \item December 2010: g=2, s=6, R=26, R\textunderscore s=22
    \item March 2011: g=2, s=4, R=24, R\textunderscore s=31
    \item July 2011: g=6, s=14, R=74, R\textunderscore s=54
    \item October 2011: g=4, s=9, R=49, R\textunderscore s=69
    \item January 2012: g=7, s=12, R=82, R\textunderscore s=97
\end{itemize}
 \item How do your estimates of R compare to the scientific estimates? Why do you think that various estimates are different?
 
 [Estimates of R should be generally increasing, similar to the scientific estimates.  Various estimates are different due to the definition of a "group" of sunspots, and how to separate sunspots in a group.]
 
 \item  After completing your five estimates, how do your estimates relate to the solar cycle graph (note the orange highlighted data points are the dates you were shown)?
 
 [Should follow a generally increasing trend in line with the data]
 
 \item Based on the solar cycle graph, approximately what is the period of the solar cycle?
 
 [11 years]
 
\end{enumerate}

\section*{Part 3: Storm Prediction}
Click on the Storm Prediction tab on this page and follow the instructions while answering the following questions: {\tt http://www.pbs.org/wgbh/nova/labs/lab/sun/research}
\begin{enumerate}
\item  What does the size of a sunspot tell us about the Sun's magnetic field and how does it help us predict solar storms?

[The larger the spot, the stronger the magnetic field -- greater potential for generating storms.]

\item  What does the complexity of sunspots tell us?

[The more complex a sunspot group (ie - more spots in the group), the more potential for a solar flare.]

\item What does rapid sunspot growth tell us about the Sun's magnetic field?

[Rapidly growing sunspots indicate a very strong magnetic field.]

\item  How does the mixing of magnetic fields help us to predict solar flares or CMEs?

[The more mixed the magnetic field (positive mixed with negative), the higher the potential for flares or CMEs.]

\item While observing the chromosphere and corona of the Sun, scientists often observe bands of plasma, called filaments. What can these filaments tell us about the possibility of a solar storm?

[Signs like bands of plasma, called filaments, which can be seen stretching across large regions in the images on the right, can indicate the strong possibility of a storm.]

\end{enumerate}


\section*{Part 4: Open Investigation}
For this section, you'll use the beta version of Helioviewer. {\tt http://beta3.helioviewer.org/}
There are a couple things that might not be immediately obvious. First, the ``measurement'' values are wavelength of the filter used to obtain the image, in Angstroms. Second, if the date/time turns red, that means there is not data of the requested type taken close to the date/time you asked for. Even if the date/time is green, it's best to check to see what time is actually being displayed, as opposed what time you input. 

I want you to generate possible investigation questions about the sun. For example, what features in the sun develop before a flare, and how do the details of those features determine the properties of the flare and its aftermath? Are the same features seen for other flares? If not, what's different about those flares? Be a scientist; explore the data for a while, ask questions, and if you have time, begin to try to answer those questions with the data. \\
\begin{enumerate}
\item Record at least 3 questions, and explain how you would investigate these questions. 
\end{enumerate}



\end{document}